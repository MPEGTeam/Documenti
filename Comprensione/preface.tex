%%!!ATTENZIONE!!
%QUESTO FILE E' UN ESEMPIO. COPIATE ALL'INTERNO DELLA CARTELLA DEL DOCUMENTO SU CUI STATE LAVORANDO E MODIFICATE
%% NON MODIFICARE QUESTE IMPOSTAZIONI
\date{}
\usepackage[latin9]{inputenc}
\usepackage{fancyhdr}
\usepackage{ucs}
\usepackage{lastpage}
\usepackage{longtable}
\usepackage[colorlinks=true]{hyperref}
\hypersetup{
    colorlinks,
    citecolor=black,
    filecolor=black,
    linkcolor=black,
    urlcolor=black
}
\pagestyle{fancy}
\fancyhead{}
\fancyfoot{}
\newcommand{\HRule}{\rule{\linewidth}{0.5mm}}
\fancyhead[RE, RO]{\doctitle - M.P.E.G.}
%%DOVETE COPIARE IL FILE LogoTeam.jpg ALL'INTERNO DELLA CARTELLA IMG (CHE DOVRETE CREARE) DENTRO LA CARTELLA DEL VOSTRO DOCUMENTO
\lhead{\setlength{\unitlength}{1mm}
	\begin{picture}(0,0)
		\put(5,0){\includegraphics[scale=0.07]{img/LogoTeam.jpg}}
	\end{picture}}
\fancyfoot[CE, CO]{\thepage\ di \pageref{LastPage}}

%%DA QUI IN POI POTETE MODIFICARE

%% INSERIRE QUI IL NOME DEL DOCUMENTO (INSERITE SEMPRE UNO SPAZIO ALLA FINE DEL NOME)
\newcommand{\doctitle}{Comprensione }

